\documentclass[landscape]{article}
\thispagestyle{empty}
\usepackage{amsmath,amssymb}
\usepackage{wasysym}
\usepackage{tikz}
\usepackage{tikz-3dplot}
\usepackage{pgfplots}
\usepackage[margin=0cm]{geometry}
   % Workaround for making use of externalization possible
   % -> remove hardcoded pdflatex and replace by lualatex
%    \usepgfplotslibrary{external}
%        \tikzset{external/system call={lualatex \tikzexternalcheckshellescape%
%         -halt-on-error -interaction=batchmode -jobname "\image" "\texsource"}}

% Redefine rotation sequence for tikz3d-plot to z-y-x
\newcommand{\tdseteulerxyz}{
\renewcommand{\tdplotcalctransformrotmain}{%
%perform some trig for the Euler transformation
\tdplotsinandcos{\sinalpha}{\cosalpha}{\tdplotalpha} 
\tdplotsinandcos{\sinbeta}{\cosbeta}{\tdplotbeta}
\tdplotsinandcos{\singamma}{\cosgamma}{\tdplotgamma}
%
\tdplotmult{\sasb}{\sinalpha}{\sinbeta}
\tdplotmult{\sasg}{\sinalpha}{\singamma}
\tdplotmult{\sasbsg}{\sasb}{\singamma}
%
\tdplotmult{\sacb}{\sinalpha}{\cosbeta}
\tdplotmult{\sacg}{\sinalpha}{\cosgamma}
\tdplotmult{\sasbcg}{\sasb}{\cosgamma}
%
\tdplotmult{\casb}{\cosalpha}{\sinbeta}
\tdplotmult{\cacb}{\cosalpha}{\cosbeta}
\tdplotmult{\cacg}{\cosalpha}{\cosgamma}
\tdplotmult{\casg}{\cosalpha}{\singamma}
%
\tdplotmult{\cbsg}{\cosbeta}{\singamma}
\tdplotmult{\cbcg}{\cosbeta}{\cosgamma}
%
\tdplotmult{\casbsg}{\casb}{\singamma}
\tdplotmult{\casbcg}{\casb}{\cosgamma}

\tdplotmult{\casbcg}{\casb}{\cosgamma}
%
%determine rotation matrix elements for Euler transformation
\pgfmathsetmacro{\raaeul}{\cbcg}
\pgfmathsetmacro{\rabeul}{-\cbsg}
\pgfmathsetmacro{\raceul}{\sinbeta }
\pgfmathsetmacro{\rbaeul}{\casg + \sasbcg}
\pgfmathsetmacro{\rbbeul}{\cacg - \sasbsg}
\pgfmathsetmacro{\rbceul}{-\sacb}
\pgfmathsetmacro{\rcaeul}{\sasg - \casbcg}
\pgfmathsetmacro{\rcbeul}{\sacg + \casbsg}
\pgfmathsetmacro{\rcceul}{\cacb}
}

% 
% \renewcommand{\tdplotcalctransformmainrot}{%
% %perform some trig for the Euler transformation
% \tdplotsinandcos{\sinalpha}{\cosalpha}{\tdplotalpha} 
% \tdplotsinandcos{\sinbeta}{\cosbeta}{\tdplotbeta}
% \tdplotsinandcos{\singamma}{\cosgamma}{\tdplotgamma}
% %
% \tdplotmult{\sasb}{\sinalpha}{\sinbeta}
% \tdplotmult{\sasg}{\sinalpha}{\singamma}
% \tdplotmult{\sasbsg}{\sasb}{\singamma}
% %
% \tdplotmult{\sacb}{\sinalpha}{\cosbeta}
% \tdplotmult{\sacg}{\sinalpha}{\cosgamma}
% \tdplotmult{\sasbcg}{\sasb}{\cosgamma}
% %
% \tdplotmult{\casb}{\cosalpha}{\sinbeta}
% \tdplotmult{\cacb}{\cosalpha}{\cosbeta}
% \tdplotmult{\cacg}{\cosalpha}{\cosgamma}
% \tdplotmult{\casg}{\cosalpha}{\singamma}
% %
% \tdplotmult{\cbsg}{\cosbeta}{\singamma}
% \tdplotmult{\cbcg}{\cosbeta}{\cosgamma}
% %
% \tdplotmult{\casbsg}{\casb}{\singamma}
% \tdplotmult{\casbcg}{\casb}{\cosgamma}
% 
% \tdplotmult{\casbcg}{\casb}{\cosgamma}
% %
% %determine rotation matrix elements for Euler transformation
% \pgfmathsetmacro{\raaeul}{\cbcg}
% \pgfmathsetmacro{\rabeul}{\casg + \sasbcg}
% \pgfmathsetmacro{\raceul}{\sasg - \casbcg }
% \pgfmathsetmacro{\rbaeul}{-\cbsg}
% \pgfmathsetmacro{\rbbeul}{\cacg - \sasbsg}
% \pgfmathsetmacro{\rbceul}{\sacg+\casbsg}
% \pgfmathsetmacro{\rcaeul}{\sinbeta}
% \pgfmathsetmacro{\rcbeul}{-\sacb}
% \pgfmathsetmacro{\rcceul}{\cacb}
% }
}

\tdseteulerxyz

\usepackage{siunitx}
 % figures
\newcommand{\sizes}{3.4} % selbstdefinierte Punktgrösse in Bilder
\newcommand{\sizem}{5.1}% selbstdefinierte Punktgrösse in Bilder

%math
\newcommand{\dd}{\textnormal{d}}
\newcommand{\dt}{\textnormal{d}t}
\newcommand{\DD}{\textnormal{D}}
\newcommand{\Dt}{\textnormal{D}t}
\newcommand{\deta}{\textnormal{d}\eta}
\newcommand{\argmin}{\mathop{\mathrm{argmin}}}
\newcommand{\Upr}{{\mathop{\mathrm{Upr}}}}
\newcommand{\Sgn}{{\mathop{\mathrm{Sgn}}}}
\newcommand{\h}{\mathop{\mathrm{H}}}
\newcommand{\prox}{{\mathop{\mathrm{prox}}}}
\newcommand{\abs}{\mathop{\mathrm{abs}}}
\newcommand{\T}{^{\mathop{\mathrm{T}}}}
\newcommand{\diag}{{\mathop{\mathrm{diag}}}}


%boldmath
%bold operator
\newcommand{\vna}{\mbox{\boldmath $\nabla$}}
%bold greek
\newcommand{\val}{\mbox{\boldmath $\alpha$}}
\newcommand{\vbe}{\mbox{\boldmath $\beta$}}
\newcommand{\vga}{\mbox{\boldmath $\gamma$}}
\newcommand{\vde}{\mbox{\boldmath $\delta$}}
\newcommand{\vep}{\mbox{\boldmath $\epsilon$}}
\newcommand{\vze}{\mbox{\boldmath $\zeta$}}
\newcommand{\vet}{\mbox{\boldmath $\eta$}}
\newcommand{\vth}{\mbox{\boldmath $\theta$}}
\newcommand{\vio}{\mbox{\boldmath $\iota$}}
\newcommand{\vka}{\mbox{\boldmath $\kappa$}}
\newcommand{\vla}{\mbox{\boldmath $\lambda$}}
\newcommand{\vmu}{\mbox{\boldmath $\mu$}}
\newcommand{\vnu}{\mbox{\boldmath $\nu$}}
\newcommand{\vxi}{\mbox{\boldmath $\xi$}}
\newcommand{\vpi}{\mbox{\boldmath $\pi$}}
\newcommand{\vrh}{\mbox{\boldmath $\rho$}}
\newcommand{\vsi}{\mbox{\boldmath $\sigma$}}
\newcommand{\vta}{\mbox{\boldmath $\tau$}}
\newcommand{\vup}{\mbox{\boldmath $\upsilon$}}
\newcommand{\vph}{\mbox{\boldmath $\varphi$}}
\newcommand{\vch}{\mbox{\boldmath $\chi$}}
\newcommand{\vps}{\mbox{\boldmath $\psi$}}
\newcommand{\vom}{\mbox{\boldmath $\omega$}}

\newcommand{\vvep}{\mbox{\boldmath $\varepsilon$}}
\newcommand{\vvth}{\mbox{\boldmath $\vartheta$}}
\newcommand{\vvrh}{\mbox{\boldmath $\varrho$}}
\newcommand{\vvpi}{\mbox{\boldmath $\varpi$}}
\newcommand{\vvsi}{\mbox{\boldmath $\varsigma$}}
\newcommand{\vvph}{\mbox{\boldmath $\phi$}}

%bold capital greek
\newcommand{\vGa}{\mathbf \Gamma}
\newcommand{\vDe}{\mathbf \Delta}
\newcommand{\vTh}{\mathbf \Theta}
\newcommand{\vLa}{\mathbf \Lambda}
\newcommand{\vXi}{\mathbf \Xi}
\newcommand{\vPi}{\mathbf \Pi}
\newcommand{\vSi}{\mathbf \Sigma}
\newcommand{\vUp}{\mathbf \Upsilon}
\newcommand{\vPh}{\mathbf \Phi}
\newcommand{\vPs}{\mathbf \Psi}
\newcommand{\vOm}{\mathbf \Omega}

%capital greek slanted, OHNE amsmath-package
%\newcommand{\iGa}{\mathnormal{\Gamma}}
%\newcommand{\iDe}{\mathnormal{\Delta}}
%\newcommand{\iTh}{\mathnormal{\Theta}}
%\newcommand{\iLa}{\mathnormal{\Lambda}}
%\newcommand{\iXi}{\mathnormal{\Xi}}
%\newcommand{\iPi}{\mathnormal{\Pi}}
%\newcommand{\iSi}{\mathnormal{\Sigma}}
%\newcommand{\iUp}{\mathnormal{\Upsilon}}
%\newcommand{\iPh}{\mathnormal{\Phi}}
%\newcommand{\iPs}{\mathnormal{\Psi}}
%\newcommand{\iOm}{\mathnormal{\Omega}}

%capital greek slanted, MIT amsmath-package
\newcommand{\iGa}{\varGamma}
\newcommand{\iDe}{\varDelta}
\newcommand{\iTh}{\varTheta}
\newcommand{\iLa}{\varLambda}
\newcommand{\iXi}{\varXi}
\newcommand{\iPi}{\varPi}
\newcommand{\iSi}{\varSigma}
\newcommand{\iUp}{\varUpsilon}
\newcommand{\iPh}{\varPhi}
\newcommand{\iPs}{\varPsi}
\newcommand{\iOm}{\varOmega}

%bold latin
\newcommand{\va}{\mathbf a}
\newcommand{\vb}{\mathbf b}
\newcommand{\vc}{\mathbf c}
\newcommand{\vd}{\mathbf d}
\newcommand{\ve}{\mathbf e}
\newcommand{\vf}{\mathbf f}
\newcommand{\vg}{\mathbf g}
\newcommand{\vh}{\mathbf h}
\newcommand{\vi}{\mathbf i}
\newcommand{\vj}{\mathbf j}
\newcommand{\vk}{\mathbf k}
\newcommand{\vl}{\mathbf l}
\newcommand{\vm}{\mathbf m}
\newcommand{\vn}{\mathbf n}
\newcommand{\vo}{\mathbf o}
\newcommand{\vp}{\mathbf p}
\newcommand{\vq}{\mathbf q}
\newcommand{\vr}{\mathbf r}
\newcommand{\vs}{\mathbf s}
\newcommand{\vt}{\mathbf t}
\newcommand{\vu}{\mathbf u}
\newcommand{\vv}{\mathbf v}
\newcommand{\vw}{\mathbf w}
\newcommand{\vx}{\mathbf x}
\newcommand{\vy}{\mathbf y}
\newcommand{\vz}{\mathbf z}
\newcommand{\eins}{\mathbf 1}

%bold capital latin
\newcommand{\vA}{\mathbf A}
\newcommand{\vB}{\mathbf B}
\newcommand{\vC}{\mathbf C}
\newcommand{\vD}{\mathbf D}
\newcommand{\vE}{\mathbf E}
\newcommand{\vF}{\mathbf F}
\newcommand{\vG}{\mathbf G}
\newcommand{\vH}{\mathbf H}
\newcommand{\vI}{\mathbf I}
\newcommand{\vJ}{\mathbf J}
\newcommand{\vK}{\mathbf K}
\newcommand{\vL}{\mathbf L}
\newcommand{\vM}{\mathbf M}
\newcommand{\vN}{\mathbf N}
\newcommand{\vO}{\mathbf O}
\newcommand{\vP}{\mathbf P}
\newcommand{\vQ}{\mathbf Q}
\newcommand{\vR}{\mathbf R}
\newcommand{\vS}{\mathbf S}
\newcommand{\vT}{\mathbf T}
\newcommand{\vU}{\mathbf U}
\newcommand{\vV}{\mathbf V}
\newcommand{\vW}{\mathbf W}
\newcommand{\vX}{\mathbf X}
\newcommand{\vY}{\mathbf Y}
\newcommand{\vZ}{\mathbf Z}

%calligraphic
\newcommand{\cA}{\mathcal{A}}
\newcommand{\cB}{\mathcal{B}}
\newcommand{\cC}{\mathcal{C}}
\newcommand{\cD}{\mathcal{D}}
\newcommand{\cE}{\mathcal{E}}
\newcommand{\cF}{\mathcal{F}}
\newcommand{\cG}{\mathcal{G}}
\newcommand{\cH}{\mathcal{H}}
\newcommand{\cI}{\mathcal{I}}
\newcommand{\cJ}{\mathcal{J}}
\newcommand{\cK}{\mathcal{K}}
\newcommand{\cL}{\mathcal{L}}
\newcommand{\cM}{\mathcal{M}}
\newcommand{\cN}{\mathcal{N}}
\newcommand{\cO}{\mathcal{O}}
\newcommand{\cP}{\mathcal{P}}
\newcommand{\cQ}{\mathcal{Q}}
\newcommand{\cR}{\mathcal{R}}
\newcommand{\cS}{\mathcal{S}}
\newcommand{\cT}{\mathcal{T}}
\newcommand{\cU}{\mathcal{U}}
\newcommand{\cV}{\mathcal{V}}
\newcommand{\cW}{\mathcal{W}}
\newcommand{\cX}{\mathcal{X}}
\newcommand{\cY}{\mathcal{Y}}
\newcommand{\cZ}{\mathcal{Z}}

%fraktur
\newcommand{\frA}{\mathfrak{A}}
\newcommand{\frB}{\mathfrak{B}}
\newcommand{\frC}{\mathfrak{C}}
\newcommand{\frD}{\mathfrak{D}}
\newcommand{\frE}{\mathfrak{E}}
\newcommand{\frF}{\mathfrak{F}}
\newcommand{\frG}{\mathfrak{G}}
\newcommand{\frH}{\mathfrak{H}}
\newcommand{\frI}{\mathfrak{I}}
\newcommand{\frJ}{\mathfrak{J}}
\newcommand{\frK}{\mathfrak{K}}
\newcommand{\frL}{\mathfrak{L}}
\newcommand{\frM}{\mathfrak{M}}
\newcommand{\frN}{\mathfrak{N}}
\newcommand{\frO}{\mathfrak{O}}
\newcommand{\frP}{\mathfrak{P}}
\newcommand{\frQ}{\mathfrak{Q}}
\newcommand{\frR}{\mathfrak{R}}
\newcommand{\frS}{\mathfrak{S}}
\newcommand{\frT}{\mathfrak{T}}
\newcommand{\frU}{\mathfrak{U}}
\newcommand{\frV}{\mathfrak{V}}
\newcommand{\frW}{\mathfrak{W}}
\newcommand{\frX}{\mathfrak{X}}
\newcommand{\frY}{\mathfrak{Y}}
\newcommand{\frZ}{\mathfrak{Z}}

\newcommand{\fra}{\mathfrak{a}}
\newcommand{\frb}{\mathfrak{b}}
\newcommand{\frc}{\mathfrak{c}}
\newcommand{\frd}{\mathfrak{d}}
\newcommand{\fre}{\mathfrak{e}}
\newcommand{\frf}{\mathfrak{f}}
\newcommand{\frg}{\mathfrak{g}}
\newcommand{\frh}{\mathfrak{h}}
\newcommand{\fri}{\mathfrak{i}}
\newcommand{\frj}{\mathfrak{j}}
\newcommand{\frk}{\mathfrak{k}}
\newcommand{\frl}{\mathfrak{l}}
\newcommand{\frm}{\mathfrak{m}}
\newcommand{\frn}{\mathfrak{n}}
\newcommand{\fro}{\mathfrak{o}}
\newcommand{\frp}{\mathfrak{p}}
\newcommand{\frq}{\mathfrak{q}}
\newcommand{\frr}{\mathfrak{r}}
\newcommand{\frs}{\mathfrak{s}}
\newcommand{\frt}{\mathfrak{t}}
\newcommand{\fru}{\mathfrak{u}}
\newcommand{\frv}{\mathfrak{v}}
\newcommand{\frw}{\mathfrak{w}}
\newcommand{\frx}{\mathfrak{x}}
\newcommand{\fry}{\mathfrak{y}}
\newcommand{\frz}{\mathfrak{z}}


%%%%%%%%% Z
\begin{document}
\clearpage
\centering



% Set the plot display orientation
% Syntax: \tdplotsetdisplay{\theta_d}{\phi_d}

 \tdplotsetmaincoords{60}{120}
%\tdplotsetmaincoords{60}{140}

\pgfmathsetmacro{\xRot}{50}
\pgfmathsetmacro{\yRot}{25}
\pgfmathsetmacro{\zRot}{30}

% Start tikz picture, and use the tdplot_main_coords style to implement the display 
% coordinate transformation provided by 3dplot.
\begin{tikzpicture}[scale=1.5,tdplot_main_coords,baseline=(current bounding box.north)]

% Set origin of main (body) coordinate system
\coordinate (O) at (0,0,0);
\node[draw=none] at (0.5,-0.8,1) {1)};
% Draw main coordinate system
\draw[black, ,->] (0,0,0) -- (1,0,0) node[anchor=south east]{$e_x^{A}$, \textcolor{red}{$e_{x'}^B$}};
\draw[black, ,->] (0,0,0) -- (0,1,0) node[anchor= west]{$e_y^{A}$};
\draw[black, ,->] (0,0,0) -- (0,0,1) node[anchor=west]{$e_z^{A}$}; 



% Intermediate frame 1
\tdplotsetrotatedcoords{\xRot}{0}{0}
\draw[thick,tdplot_rotated_coords,->, red] (0,0,0) -- (1,0,0) node[anchor=west]{};
\draw[tdplot_rotated_coords,->, red] (0,0,0) -- (0,1,0) node[anchor=west]{$e_{y'}^B$};
\draw[tdplot_rotated_coords,->, red] (0,0,0) -- (0,0,1) node[anchor=west]{$e_{z'}^B$};

\tdplotsetrotatedcoords{0}{90}{0}
\tdplotdrawarc[thick,tdplot_rotated_coords,->,color=gray]{(0,0,0)}{.5}{180}{180+\xRot}{anchor=south,color=gray}{$\alpha$}

\end{tikzpicture}
%%%%%%%%%%%%%%%%% Z-Y
\begin{tikzpicture}[scale=1.5,tdplot_main_coords,font=\small,baseline=(current bounding box.north)] 
        thick,
% Set origin of main (body) coordinate system
\coordinate (O) at (0,0,0);
\node[draw=none] at (0.5,-0.8,1) {2)};
% Draw main coordinate system
\draw[black, ,->] (0,0,0) -- (1,0,0) node[anchor=south east]{$e_x^{A}$, \textcolor{red}{$e_{x'}^B$}};
\draw[black, ,->] (0,0,0) -- (0,1,0) node[anchor= west]{$e_y^{A}$};
\draw[black, ,->] (0,0,0) -- (0,0,1) node[anchor=west]{$e_z^{A}$}; 

% Intermediate frame 1
\tdplotsetrotatedcoords{\xRot}{0}{0}
\draw[tdplot_rotated_coords,->, red] (0,0,0) -- (1,0,0) node[anchor=west]{};
\draw[tdplot_rotated_coords,->, red] (0,0,0) -- (0,1,0) node[anchor=west]{{$e_{y'}^B$}, \textcolor{green!50!black}{$e_{y''}^C$} };
\draw[tdplot_rotated_coords,->, red] (0,0,0) -- (0,0,1) node[anchor=west]{$e_{z'}^B$};

\tdplotsetrotatedthetaplanecoords{0}
\tdplotdrawarc[thick,tdplot_rotated_coords,->,color=gray]{(0,0,0)}{0.5}{90}{90+\yRot}{anchor=north east,color=gray}{$\beta$}


\tdplotsetrotatedcoords{0}{90}{0}
\tdplotdrawarc[thick,tdplot_rotated_coords,->,color=gray]{(0,0,0)}{.5}{180}{180+\xRot}{anchor=south,color=gray}{$\alpha$}

\tdplotsetrotatedcoords{0}{0}{0}
% Intermediate frame 2
\tdplotsetrotatedcoords{\xRot}{\yRot}{0}
\draw[tdplot_rotated_coords,->, green!50!black] (0,0,0) -- (1,0,0)
node[anchor=west ]{$e_{x''}^C$};
\draw[thick,tdplot_rotated_coords,->, green!50!black] (0,0,0) -- (0,1,0)
node[anchor=south west]{};
\draw[tdplot_rotated_coords,->, green!50!black] (0,0,0) -- (0,0,1)
node[anchor=north ]{$e_{z''}^C$};




\end{tikzpicture}
%%%%%%%%%%%%% Z-Y-X
\begin{tikzpicture}[scale=1.5,tdplot_main_coords,baseline=(current bounding box.north)]
\node(rectangle) at (0,5.5,1.6){$\tiny\begin{aligned}
\vC_{D\!A} &= {\color{blue}\vC_{D\!C}} {\color{green!50!black}\vC_{C\!B}} {\color{red}\vC_{B\!A}} \quad \Rightarrow \quad _{D}\vr = \vC_{D\!A} {}_{A}\vr \\
         &= {\color{blue}\begin{pmatrix} \cos{\alpha} & \sin{\alpha} & 0 \\ -\sin{\alpha} & \cos{\alpha} & 0 \\ 0 & 0 & 1 \end{pmatrix}} {\color{green!50!black}\begin{pmatrix} \cos{\beta} & 0 & -\sin{\beta} \\ 0 & 1 & 0 \\ \sin{\beta} & 0 & \cos{\beta} \end{pmatrix}}  {\color{red}\begin{pmatrix} 1 & 0 & 0 \\ 0 & \cos{\alpha} & \sin{\alpha} \\ 0 & -\sin{\alpha} & \cos{\alpha} \end{pmatrix}}\\
	&=  \begin{pmatrix} \cos{\beta}\cos{\gamma} & \cos{\alpha}\sin{\gamma} + \cos{\gamma}\sin{\alpha}\sin{\beta} & \sin{\alpha}\sin{\gamma} - \cos{\alpha}\cos{\gamma}\sin{\beta} \\
 -\cos{\beta}\sin{\gamma} & \cos{\alpha}\cos{\gamma} - \sin{\alpha}\sin{\beta}\sin{\gamma} & \cos{\gamma}\sin{\alpha} + \cos{\alpha}\sin{\beta}\sin{\gamma} \\
             \sin{\beta} &                                   -\cos{\beta}\sin{\alpha} &                                    \cos{\alpha}\cos{\beta} \\
 \end{pmatrix} \\
\vC_{A\!D} &= \vC_{D\!A}^\mathsf{T} \\
&=  \begin{pmatrix}                                \cos{\beta}\cos{\gamma} &                                   -\cos{\beta}\sin{\gamma} &             \sin{\beta} \\
 \cos{\alpha}\sin{\gamma} + \cos{\gamma}\sin{\alpha}\sin{\beta} & \cos{\alpha}\cos{\gamma} - \sin{\alpha}\sin{\beta}\sin{\gamma} & -\cos{\beta}\sin{\alpha} \\
 \sin{\alpha}\sin{\gamma} - \cos{\alpha}\cos{\gamma}\sin{\beta} & \cos{\gamma}\sin{\alpha} + \cos{\alpha}\sin{\beta}\sin{\gamma} &  \cos{\alpha}\cos{\beta} \\
\end{pmatrix} \\
\end{aligned}$};
% Set origin of main (body) coordinate system
\coordinate (O) at (0,0,0);
\node[draw=none] at (0.5,-0.8,1) {3)};

% Draw main coordinate system
\draw[black, ,->] (0,0,0) -- (1,0,0) node[anchor=south east]{$e_x^{A}$, \textcolor{red}{$e_{x'}^B$}};
\draw[black, ,->] (0,0,0) -- (0,1,0) node[anchor= west]{$e_y^{A}$};
\draw[black, ,->] (0,0,0) -- (0,0,1) node[anchor=west]{$e_z^{A}$}; 



% Intermediate frame 1
\tdplotsetrotatedcoords{\xRot}{0}{0}
\draw[tdplot_rotated_coords,->, red] (0,0,0) -- (1,0,0) node[anchor=west]{};
\draw[tdplot_rotated_coords,->, red] (0,0,0) -- (0,1,0) node[anchor=north west]{{$e_{y'}^B$}, \textcolor{green!50!black}{$e_{y''}^C$}};
\draw[tdplot_rotated_coords,->, red] (0,0,0) -- (0,0,1) node[anchor=west]{$e_{z'}^B$};

\tdplotsetrotatedthetaplanecoords{0}
\tdplotdrawarc[thick,tdplot_rotated_coords,->,color=gray]{(0,0,0)}{0.5}{90}{90+\yRot}{anchor=north east,color=gray}{$\beta$}

\tdplotsetrotatedcoords{0}{90}{0}
\tdplotdrawarc[thick,tdplot_rotated_coords,->,color=gray]{(0,0,0)}{.5}{180}{180+\xRot}{anchor=south,color=gray}{$\alpha$}

% Intermediate frame 2
\tdplotsetrotatedcoords{0}{0}{0}
% Intermediate frame 2
\tdplotsetrotatedcoords{\xRot}{\yRot}{0}
\draw[tdplot_rotated_coords,->, green!50!black] (0,0,0) -- (1,0,0)
node[anchor=west ]{$e_{x''}^C$};
\draw[tdplot_rotated_coords,->, green!50!black] (0,0,0) -- (0,1,0)
node[anchor=south]{};
\draw[tdplot_rotated_coords,->, green!50!black] (0,0,0) -- (0,0,1)
node[anchor=north east]{$e_{z''}^C$, \textcolor{blue}{$e_{z'''}^D$}};

%\tdplotsetrotatedcoords{\zRot}{\yRot+90}{0}
\tdplotdrawarc[thick,tdplot_rotated_coords,->,color=gray]{(0,0,0)}{0.5}{90}{90+\zRot}{anchor=north west,color=gray}{$\gamma$}

\tdplotsetrotatedcoords{0}{0}{0}
% Rotate to final frame
\tdplotsetrotatedcoords{\xRot}{\yRot}{\zRot}
\draw[tdplot_rotated_coords,->, blue] (0,0,0) -- (1,0,0)
node[anchor=west]{$e_{x'''}^D$};
\draw[tdplot_rotated_coords,->, blue] (0,0,0) -- (0,1,0) node[anchor=west ]{$e_{y'''}^D$};
\draw[thick,tdplot_rotated_coords,->, blue] (0,0,0) -- (0,0,1) node[anchor=north east]{};






\end{tikzpicture}


\end{document}
