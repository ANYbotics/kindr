\documentclass[10pt,landscape,a4paper]{article}
\usepackage{nonfloat}
\usepackage{multicol}
\usepackage[font=scriptsize]{caption}
\usepackage{calc}
\usepackage{ifthen}
\usepackage[landscape]{geometry}
\usepackage{amsmath}
\usepackage{amssymb}
\usepackage{multirow}
 \usepackage[T1]{fontenc}
% \usepackage{bbold}
    \usepackage{dsfont}
%\usepackage{esvect}
\usepackage{graphics}
   \usepackage[pdftex]{graphicx}

\usepackage{tabularx,booktabs} 
\usepackage{epstopdf}
 \usepackage{ifpdf}
\usepackage{sectsty}
\usepackage{tikz}
\usepackage{mathtools} % \mathclap for underbrace
\paragraphfont{\small}

\usepackage[explicit]{titlesec}
\usepackage{xcolor}

% To make this come out properly in landscape mode, do one of the following
% 1.
%  pdflatex latexsheet.tex
%
% 2.
%  latex latexsheet.tex
%  dvips -P pdf  -t landscape latexsheet.dvi
%  ps2pdf latexsheet.ps


% If you're reading this, be prepared for confusion.  Making this was
% a learning experience for me, and it shows.  Much of the placement
% was hacked in; if you make it better, let me know...


% 2008-04
% Changed page margin code to use the geometry package. Also added code for
% conditional page margins, depending on paper size. Thanks to Uwe Ziegenhagen
% for the suggestions.

% 2006-08
% Made changes based on suggestions from Gene Cooperman. <gene at ccs.neu.edu>


% To Do:
% \listoffigures \listoftables
% \setcounter{secnumdepth}{0}


% This sets page margins to .5 inch if using letter paper, and to 1cm
% if using A4 paper. (This probably isn't strictly necessary.)
% If using another size paper, use default 1cm margins.
\ifthenelse{\lengthtest { \paperwidth = 11in}}
	{ \geometry{top=.5in,left=.5in,right=.5in,bottom=.5in} }
	{\ifthenelse{ \lengthtest{ \paperwidth = 297mm}}
		{\geometry{top=1cm,left=1cm,right=1cm,bottom=1cm} }
		{\geometry{top=1cm,left=1cm,right=1cm,bottom=1cm} }
	}

% Turn off header and footer
\pagestyle{empty}
 

% Redefine section commands to use less space
\makeatletter
  \renewcommand{\section}{\@startsection{section}{1}{0mm}%
                               {-1ex plus -.5ex minus -.2ex}%
                                {0.5ex plus .2ex}%x
                                {\normalfont\large\bfseries}}
  
\renewcommand{\subsection}{\@startsection{subsection}{2}{0mm}%
                                {-1explus -.5ex minus -.2ex}%
                                {0.5ex plus .2ex}%
                                {\normalfont\normalsize\bfseries}}
\renewcommand{\subsubsection}{\@startsection{subsubsection}{3}{0mm}%
                                {-1ex plus -.5ex minus -.2ex}%
                                {1ex plus .2ex}%
                                {\normalfont\small\bfseries}}
\makeatother

% Define BibTeX command
\def\BibTeX{{\rm B\kern-.05em{\sc i\kern-.025em b}\kern-.08em
    T\kern-.1667em\lower.7ex\hbox{E}\kern-.125emX}}

% Don't print section numbers
\setcounter{secnumdepth}{0}


\setlength{\parindent}{0pt}
\setlength{\parskip}{0pt plus 0.5ex}


 % figures
\newcommand{\sizes}{3.4} % selbstdefinierte Punktgrösse in Bilder
\newcommand{\sizem}{5.1}% selbstdefinierte Punktgrösse in Bilder

%math
\newcommand{\dd}{\textnormal{d}}
\newcommand{\dt}{\textnormal{d}t}
\newcommand{\DD}{\textnormal{D}}
\newcommand{\Dt}{\textnormal{D}t}
\newcommand{\deta}{\textnormal{d}\eta}
\newcommand{\argmin}{\mathop{\mathrm{argmin}}}
\newcommand{\Upr}{{\mathop{\mathrm{Upr}}}}
\newcommand{\Sgn}{{\mathop{\mathrm{Sgn}}}}
\newcommand{\h}{\mathop{\mathrm{H}}}
\newcommand{\prox}{{\mathop{\mathrm{prox}}}}
\newcommand{\abs}{\mathop{\mathrm{abs}}}
\newcommand{\T}{^{\mathop{\mathrm{T}}}}
\newcommand{\diag}{{\mathop{\mathrm{diag}}}}


%boldmath
%bold operator
\newcommand{\vna}{\mbox{\boldmath $\nabla$}}
%bold greek
\newcommand{\val}{\mbox{\boldmath $\alpha$}}
\newcommand{\vbe}{\mbox{\boldmath $\beta$}}
\newcommand{\vga}{\mbox{\boldmath $\gamma$}}
\newcommand{\vde}{\mbox{\boldmath $\delta$}}
\newcommand{\vep}{\mbox{\boldmath $\epsilon$}}
\newcommand{\vze}{\mbox{\boldmath $\zeta$}}
\newcommand{\vet}{\mbox{\boldmath $\eta$}}
\newcommand{\vth}{\mbox{\boldmath $\theta$}}
\newcommand{\vio}{\mbox{\boldmath $\iota$}}
\newcommand{\vka}{\mbox{\boldmath $\kappa$}}
\newcommand{\vla}{\mbox{\boldmath $\lambda$}}
\newcommand{\vmu}{\mbox{\boldmath $\mu$}}
\newcommand{\vnu}{\mbox{\boldmath $\nu$}}
\newcommand{\vxi}{\mbox{\boldmath $\xi$}}
\newcommand{\vpi}{\mbox{\boldmath $\pi$}}
\newcommand{\vrh}{\mbox{\boldmath $\rho$}}
\newcommand{\vsi}{\mbox{\boldmath $\sigma$}}
\newcommand{\vta}{\mbox{\boldmath $\tau$}}
\newcommand{\vup}{\mbox{\boldmath $\upsilon$}}
\newcommand{\vph}{\mbox{\boldmath $\varphi$}}
\newcommand{\vch}{\mbox{\boldmath $\chi$}}
\newcommand{\vps}{\mbox{\boldmath $\psi$}}
\newcommand{\vom}{\mbox{\boldmath $\omega$}}

\newcommand{\vvep}{\mbox{\boldmath $\varepsilon$}}
\newcommand{\vvth}{\mbox{\boldmath $\vartheta$}}
\newcommand{\vvrh}{\mbox{\boldmath $\varrho$}}
\newcommand{\vvpi}{\mbox{\boldmath $\varpi$}}
\newcommand{\vvsi}{\mbox{\boldmath $\varsigma$}}
\newcommand{\vvph}{\mbox{\boldmath $\phi$}}

%bold capital greek
\newcommand{\vGa}{\mathbf \Gamma}
\newcommand{\vDe}{\mathbf \Delta}
\newcommand{\vTh}{\mathbf \Theta}
\newcommand{\vLa}{\mathbf \Lambda}
\newcommand{\vXi}{\mathbf \Xi}
\newcommand{\vPi}{\mathbf \Pi}
\newcommand{\vSi}{\mathbf \Sigma}
\newcommand{\vUp}{\mathbf \Upsilon}
\newcommand{\vPh}{\mathbf \Phi}
\newcommand{\vPs}{\mathbf \Psi}
\newcommand{\vOm}{\mathbf \Omega}

%capital greek slanted, OHNE amsmath-package
%\newcommand{\iGa}{\mathnormal{\Gamma}}
%\newcommand{\iDe}{\mathnormal{\Delta}}
%\newcommand{\iTh}{\mathnormal{\Theta}}
%\newcommand{\iLa}{\mathnormal{\Lambda}}
%\newcommand{\iXi}{\mathnormal{\Xi}}
%\newcommand{\iPi}{\mathnormal{\Pi}}
%\newcommand{\iSi}{\mathnormal{\Sigma}}
%\newcommand{\iUp}{\mathnormal{\Upsilon}}
%\newcommand{\iPh}{\mathnormal{\Phi}}
%\newcommand{\iPs}{\mathnormal{\Psi}}
%\newcommand{\iOm}{\mathnormal{\Omega}}

%capital greek slanted, MIT amsmath-package
\newcommand{\iGa}{\varGamma}
\newcommand{\iDe}{\varDelta}
\newcommand{\iTh}{\varTheta}
\newcommand{\iLa}{\varLambda}
\newcommand{\iXi}{\varXi}
\newcommand{\iPi}{\varPi}
\newcommand{\iSi}{\varSigma}
\newcommand{\iUp}{\varUpsilon}
\newcommand{\iPh}{\varPhi}
\newcommand{\iPs}{\varPsi}
\newcommand{\iOm}{\varOmega}

%bold latin
\newcommand{\va}{\mathbf a}
\newcommand{\vb}{\mathbf b}
\newcommand{\vc}{\mathbf c}
\newcommand{\vd}{\mathbf d}
\newcommand{\ve}{\mathbf e}
\newcommand{\vf}{\mathbf f}
\newcommand{\vg}{\mathbf g}
\newcommand{\vh}{\mathbf h}
\newcommand{\vi}{\mathbf i}
\newcommand{\vj}{\mathbf j}
\newcommand{\vk}{\mathbf k}
\newcommand{\vl}{\mathbf l}
\newcommand{\vm}{\mathbf m}
\newcommand{\vn}{\mathbf n}
\newcommand{\vo}{\mathbf o}
\newcommand{\vp}{\mathbf p}
\newcommand{\vq}{\mathbf q}
\newcommand{\vr}{\mathbf r}
\newcommand{\vs}{\mathbf s}
\newcommand{\vt}{\mathbf t}
\newcommand{\vu}{\mathbf u}
\newcommand{\vv}{\mathbf v}
\newcommand{\vw}{\mathbf w}
\newcommand{\vx}{\mathbf x}
\newcommand{\vy}{\mathbf y}
\newcommand{\vz}{\mathbf z}
\newcommand{\eins}{\mathbf 1}

%bold capital latin
\newcommand{\vA}{\mathbf A}
\newcommand{\vB}{\mathbf B}
\newcommand{\vC}{\mathbf C}
\newcommand{\vD}{\mathbf D}
\newcommand{\vE}{\mathbf E}
\newcommand{\vF}{\mathbf F}
\newcommand{\vG}{\mathbf G}
\newcommand{\vH}{\mathbf H}
\newcommand{\vI}{\mathbf I}
\newcommand{\vJ}{\mathbf J}
\newcommand{\vK}{\mathbf K}
\newcommand{\vL}{\mathbf L}
\newcommand{\vM}{\mathbf M}
\newcommand{\vN}{\mathbf N}
\newcommand{\vO}{\mathbf O}
\newcommand{\vP}{\mathbf P}
\newcommand{\vQ}{\mathbf Q}
\newcommand{\vR}{\mathbf R}
\newcommand{\vS}{\mathbf S}
\newcommand{\vT}{\mathbf T}
\newcommand{\vU}{\mathbf U}
\newcommand{\vV}{\mathbf V}
\newcommand{\vW}{\mathbf W}
\newcommand{\vX}{\mathbf X}
\newcommand{\vY}{\mathbf Y}
\newcommand{\vZ}{\mathbf Z}

%calligraphic
\newcommand{\cA}{\mathcal{A}}
\newcommand{\cB}{\mathcal{B}}
\newcommand{\cC}{\mathcal{C}}
\newcommand{\cD}{\mathcal{D}}
\newcommand{\cE}{\mathcal{E}}
\newcommand{\cF}{\mathcal{F}}
\newcommand{\cG}{\mathcal{G}}
\newcommand{\cH}{\mathcal{H}}
\newcommand{\cI}{\mathcal{I}}
\newcommand{\cJ}{\mathcal{J}}
\newcommand{\cK}{\mathcal{K}}
\newcommand{\cL}{\mathcal{L}}
\newcommand{\cM}{\mathcal{M}}
\newcommand{\cN}{\mathcal{N}}
\newcommand{\cO}{\mathcal{O}}
\newcommand{\cP}{\mathcal{P}}
\newcommand{\cQ}{\mathcal{Q}}
\newcommand{\cR}{\mathcal{R}}
\newcommand{\cS}{\mathcal{S}}
\newcommand{\cT}{\mathcal{T}}
\newcommand{\cU}{\mathcal{U}}
\newcommand{\cV}{\mathcal{V}}
\newcommand{\cW}{\mathcal{W}}
\newcommand{\cX}{\mathcal{X}}
\newcommand{\cY}{\mathcal{Y}}
\newcommand{\cZ}{\mathcal{Z}}

%fraktur
\newcommand{\frA}{\mathfrak{A}}
\newcommand{\frB}{\mathfrak{B}}
\newcommand{\frC}{\mathfrak{C}}
\newcommand{\frD}{\mathfrak{D}}
\newcommand{\frE}{\mathfrak{E}}
\newcommand{\frF}{\mathfrak{F}}
\newcommand{\frG}{\mathfrak{G}}
\newcommand{\frH}{\mathfrak{H}}
\newcommand{\frI}{\mathfrak{I}}
\newcommand{\frJ}{\mathfrak{J}}
\newcommand{\frK}{\mathfrak{K}}
\newcommand{\frL}{\mathfrak{L}}
\newcommand{\frM}{\mathfrak{M}}
\newcommand{\frN}{\mathfrak{N}}
\newcommand{\frO}{\mathfrak{O}}
\newcommand{\frP}{\mathfrak{P}}
\newcommand{\frQ}{\mathfrak{Q}}
\newcommand{\frR}{\mathfrak{R}}
\newcommand{\frS}{\mathfrak{S}}
\newcommand{\frT}{\mathfrak{T}}
\newcommand{\frU}{\mathfrak{U}}
\newcommand{\frV}{\mathfrak{V}}
\newcommand{\frW}{\mathfrak{W}}
\newcommand{\frX}{\mathfrak{X}}
\newcommand{\frY}{\mathfrak{Y}}
\newcommand{\frZ}{\mathfrak{Z}}

\newcommand{\fra}{\mathfrak{a}}
\newcommand{\frb}{\mathfrak{b}}
\newcommand{\frc}{\mathfrak{c}}
\newcommand{\frd}{\mathfrak{d}}
\newcommand{\fre}{\mathfrak{e}}
\newcommand{\frf}{\mathfrak{f}}
\newcommand{\frg}{\mathfrak{g}}
\newcommand{\frh}{\mathfrak{h}}
\newcommand{\fri}{\mathfrak{i}}
\newcommand{\frj}{\mathfrak{j}}
\newcommand{\frk}{\mathfrak{k}}
\newcommand{\frl}{\mathfrak{l}}
\newcommand{\frm}{\mathfrak{m}}
\newcommand{\frn}{\mathfrak{n}}
\newcommand{\fro}{\mathfrak{o}}
\newcommand{\frp}{\mathfrak{p}}
\newcommand{\frq}{\mathfrak{q}}
\newcommand{\frr}{\mathfrak{r}}
\newcommand{\frs}{\mathfrak{s}}
\newcommand{\frt}{\mathfrak{t}}
\newcommand{\fru}{\mathfrak{u}}
\newcommand{\frv}{\mathfrak{v}}
\newcommand{\frw}{\mathfrak{w}}
\newcommand{\frx}{\mathfrak{x}}
\newcommand{\fry}{\mathfrak{y}}
\newcommand{\frz}{\mathfrak{z}}

% -----------------------------------------------------------------------
\DeclareGraphicsExtensions{.pdf,.eps}

\DeclareMathOperator{\tr}{tr}
\DeclareMathOperator{\logmap}{log}
\DeclareMathOperator{\expmap}{exp}
\DeclareMathOperator{\logmatrix}{logM}
\DeclareMathOperator{\expmatrix}{expM}
% identity matrix
\newcommand\identity{\mathds{1}}
\newcommand\zero{\mathbf{0}}
\newcommand\norm[1]{\lVert #1 \rVert}
\newcommand\transpose{\mathsf{T}}

%\newcommand\imquatvec[1]{\overrightarrow{\mathbf{#1}}}
\newcommand\imquatvec[1]{\check{\mathbf{#1}}}

\newcommand\pos[3]{{}_#1\vr_{#2\!#3}}
\newcommand\postranspose[3]{{}_#1\vr_{#2\!#3}^\transpose}
\newcommand\rotmat[2]{\vR_{#1\!#2}}
\newcommand\drotmat[2]{\dot{\vR}_{#1\!#2}}
\newcommand\comat[2]{\vC_{#1\!#2}}
\newcommand\dcomat[2]{\dot{\vC}_{#1\!#2}}
\newcommand\quat[2]{\vq_{#1\!#2}}
\newcommand\Quat[2]{Q_{#1\!#2}}
\newcommand\angleaxis[2]{(\theta,\vn)_{#1\!#2}}
\newcommand\rotvec[2]{\vvph_{#1\!#2}}
\newcommand\linvel[2]{{}_#1\vv_{#2}}
\newcommand\rotvel[3]{{}_#1\vom_{#2\!#3}}
\newcommand\rotvelhat[3]{{}_#1\hat{\vom}_{#2\!#3}}

\newcommand\myfigure[1]{%
\medskip\noindent\begin{minipage}{\columnwidth}
\centering%
#1%
%figure,caption, and label go here
\end{minipage}\medskip}

\begin{document}

\raggedright
\footnotesize
\begin{multicols}{2}

% multicol parameters
% These lengths are set only within the two main columns
%\setlength{\columnseprule}{0.25pt}
\setlength{\premulticols}{1pt}
\setlength{\postmulticols}{1pt}
\setlength{\multicolsep}{1pt}
\setlength{\columnsep}{2pt}

\begin{center}
     \Large{\textbf{Kindr Library}} \small{\textbf{-- Kinematics and Dynamics for Robotics}}\\
       %\small{\\\textbf{Kinematics and Dynamics for Robotics}}}  \\
        \vspace{2mm}\scriptsize{Christian Gehring, C. Dario Bellicoso, Michael Bloesch, Hannes Sommer, Peter Fankhauser, \\ Marco Hutter, Roland Siegwart} \\ 
        %\scriptsize{Autonomous Systems Lab & Robotics Systems Lab, ETH Zurich}
\end{center}
%%%%%%%%%%%%%%%%%%%%%%%%%%%%%%%%%%%%%%%%%%%%%%%%%%%%%%%%%%%%%%%%%%%%%%%%%%%%%%%%%%%%%%%
%  BEGIN CONTENT
%%%%%%%%%%%%%%%%%%%%%%%%%%%%%%%%%%%%%%%%%%%%%%%%%%%%%%%%%%%%%%%%%%%%%%%%%%%%%%%%%%%%%%%
\section{Nomenclature}
\begin{tabular}{ll@{   }l}
\hline
(Hyper-)complex number & $Q$ & normal capital letter  \\ \hline
Column vector & $\va$ & bold small letter  \\ \hline
Matrix & $\vM$ & bold capital letter  \\ \hline
Identity matrix & $\identity_{n\times m}$ & ${n \times m}$-matrix  \\  \hline
Coordinate system (CS) & ${\ve_x^A,\ve_y^A,\ve_z^A}$ & Cartesian right-hand system $A$ with basis (unit) vectors $\ve$  \\ \hline
Inertial frame & ${\ve_x^I,\ve_y^I,\ve_z^I}$ & global / inertial / world coordinate system (never moves) \\ \hline
Body-fixed frame & ${\ve_x^B,\ve_y^B,\ve_z^B}$ & local / body-fixed coordinate system (moves with body) \\ \hline
Rotation & $\Phi \in \mathrm{SO}(3)$ & generic rotation (for all parameterizations) \\ \hline
Machine precision & $\epsilon$ & \\ \hline

\end{tabular}
\section{Operators}
\begin{tabular}{ll}
\hline
Cross product & $\va \times \vb = \begin{bmatrix} a_1 \\ a_2 \\ a_3\end{bmatrix} \times \begin{bmatrix} b_1 \\ b_2 \\ b_3\end{bmatrix} \Leftrightarrow (\va)^\wedge \vb = \hat{\va}\vb = \begin{bmatrix} 0 & -a_3 & a_2 \\ a_3 & 0 & -a_1 \\ -a_2 & a_1 & 0 \end{bmatrix} \begin{bmatrix} b_1 \\ b_2 \\ b_3\end{bmatrix}$ \\ \hline
Skew/unskew &  $ \va = \hat{\va}^\vee$ \\ \hline
Euclidean norm & $\norm{\va} = \sqrt{\va^T\va} = \sqrt{a_1^2 + \ldots + a_n^2}$ \\ \hline
Exponential map for matrix & $\expmatrix: \mathbb{R}^3 \rightarrow \mathbb{R}^3,  \vA \mapsto e^\vA, \quad \vA \in \mathbb{R}^{3\times 3}$ \\ \hline
Logarithmic map for matrix & $\logmatrix: \mathbb{R}^3 \rightarrow \mathbb{R}^3, \vA \mapsto \log{\vA}, \quad  \vA \in \mathbb{R}^{3\times 3}$\\ \hline
\end{tabular}

\section{Position \& Orientation}
\subsection{Position}
\begin{tabular}{lll}
 \hline
Vector & $\vr_{O\!P}$ & from point $O$ to point $P$ \\ \hline
Position vector & $\pos{B}{O}{P} \in \mathbb{R}^3 $ & from point $O$ to point $P$ expr. in frame $B$ \\ \hline 
Homogeneous pos. vector & ${}_B\bar{\vr}_{O\!P} = \begin{bmatrix}\postranspose{B}{O}{P} & 1 \end{bmatrix}^\transpose$ & from point $O$ to point $P$ expr. in frame $B$ \\ \hline
\end{tabular}

\subsection{Orientation/Rotation}
\parbox[t]{0.7\columnwidth}{\null
  \vskip-\abovecaptionskip
\begin{tabular}[h!]{r@{ }p{3cm}@{ }l@{ }}
 1) & Active Rotation: & $\Phi^A: {}_I\color{blue}{\vr_{O\!P}} \color{black}\mapsto {}_I\color{green!50!black}{\vr_{O\!Q}} $ (rotates the vector $\vr_{O\!P}$) \\
 2) & Passive Rotation:& $\Phi^P: {}_I\color{blue}{\vr_{O\!P}} \color{black}\mapsto \color{red}{}_B\color{blue}{\vr_{O\!P}} $ (rotates the frame $(\ve_x^I,\ve_y^I,\ve_z^I)$)\\
 3) & Elementary Rotations &  ${}_I\color{blue}{\vr_{O\!P}} = \color{black} \comat{I}{B} \color{red}{}_B\color{blue}{\vr_{O\!P}}$ \\ 
 & & around z-axis: \tiny  $\comat{I}{B} = \begin{bmatrix} \cos{\theta} & -\sin{\theta} & 0 \\ 
			    \sin{\theta} & \cos{\theta} & 0 \\
			     0           & 0 & 1  
			     \end{bmatrix}$ \\
 & & around y-axis: \tiny  $\comat{I}{B} = \begin{bmatrix} \cos{\theta} & 0 & \sin{\theta} \\ 
			     0 & 1 & 0 \\
			     -\sin{\theta}          & 0 & \cos{\theta}  
			     \end{bmatrix}$ \\
 & & around x-axis: \tiny  $\comat{I}{B} = \begin{bmatrix} 1 & 0            & 0 \\ 
							   0 & \cos{\theta} & -\sin{\theta} \\
							   0 & \sin{\theta} & \cos{\theta} 
					    \end{bmatrix}$ \\
 4) & Inversion: & $\Phi^{A^{-1}}(\vr) = \Phi^P(\vr)$ \\ 
 5) & Concatenation: & \multicolumn{1}{l}{$\begin{aligned}\Phi_2^A\left(\Phi_1^A(\vr)\right) &= \left(\Phi_2^A \otimes \Phi_1^A\right)(\vr) = \left(\Phi_1^{A^{-1}} \otimes \Phi_2^{A^{-1}}\right)^{-1}(\vr) \\
\Phi_2^P\left(\Phi_1^P(\vr)\right) &= \left(\Phi_2^P \otimes \Phi_1^P\right)(\vr) 
									     = \left(\Phi_1^{P^{-1}} \otimes \Phi_2^{P^{-1}}\right)^{-1}(\vr) 
					      \end{aligned}$} \\
 6) & Exponential map: & $ \expmap: \mathbb{R}^3 \rightarrow \mathrm{SO}(3), \vv \mapsto \expmatrix(\hat{\vv}), \quad \vv \in \mathbb{R}^3$ \\
 7) & Logarithmic map: & $ \logmap: \mathrm{SO}(3) \rightarrow \mathbb{R}^3, \Phi \mapsto \logmatrix(\Phi)^\vee, \quad \Phi \in \mathrm{SO}(3)$ \\ 
 8) & Box plus: & $\Phi_{2} = \Phi_{1} \boxplus \vv = \expmap{(\vv)} \otimes \Phi_{1}, \quad  \Phi_1, \Phi_2 \in \mathrm{SO}(3), \vv \in \mathbb{R}^3$   \\
 9) & Box minus: & $\vv = \Phi_{1} \boxminus \Phi_2 = \logmap{(\Phi_1\otimes\Phi_2^{-1})}, \quad \Phi_1, \Phi_2 \in \mathrm{SO}(3), \vv \in \mathbb{R}^3$  \\ 
 10) & Discrete integration: & $\Phi_{I\!B}^{k+1} = \Phi_{I\!B}^k \boxplus ({}_I\vom_{I\!B}^k\Delta t), \quad \Phi_{B\!I}^{k+1} = \Phi_{B\!I}^k \boxplus (-{}_B\vom_{I\!B}^k\Delta t)$ \\
 11) & Discrete differential: & ${}_I\vom_{I\!B}^k = (\Phi_{I\!B}^{k+1} \boxminus \Phi_{I\!B}^k)/\Delta t, \quad {}_B\vom_{I\!B}^k = -(\Phi_{B\!I}^{k+1} \boxminus{\Phi_{B\!I}^k }/\Delta t)$ \\
 12) & (Spherical) linear interpolation $t \in [0, 1]$: & $\begin{aligned}\Phi_t &= \Phi_0 \boxplus \left( (\Phi_1 \boxminus \Phi_0) t \right), \quad \Phi_t = \Phi(t), \Phi_0 = \Phi(0), \Phi_1=\Phi(1) \\ &= (\Phi_1 \otimes \Phi_0^{-1})^t \otimes \Phi_0 \end{aligned}$ \\ 
\end{tabular}
}
\parbox[t]{0.25\columnwidth}{\null
\vspace{+3mm}
        \includegraphics[width=0.25\columnwidth]{coordinate_systems/rotation_active_passive-crop.pdf}
}
\subsubsection{Rotation Parameterizations}
\begin{tabular}{ll@{}l@{}}
\hline
Rotation Matrix& $\comat{A}{B} \in \mathrm{SO}(3)$ & The rotation matrix (Direction Cosine Matrix)  \\  
  & $ \pos{A}{O}{P} = \comat{A}{B} \pos{B}{O}{P}$ & is a coordinate transformation matrix,  \\
& $\comat{A}{B} = \comat{B}{A}^\transpose$ & which transforms vectors from frame $B$ to frame $A$. \\  \hline 
 Rotation  & $\quat{A}{B}$ &  Hamiltonian unit quaternion (hypercomplex number)\\
Quaternion &  $\vq = [ q_0 \, q_1  \, q_2  \,  q_3 ]^\transpose $ & $Q = q_0 + q_1 i + q_2 j + q_3 k \in \mathbb{H}, \quad q_i \in \mathbb{R}, \quad \norm{Q}= 1$ \\ \hline
Angle-axis & $\angleaxis{A}{B}$ &   Rotation with unit rotation axis $\vn$ and angle $\theta \in [0, \pi]$. \\ \hline
Rotation Vector & $ \rotvec{A}{B} $  &  Rotation with rotation axis $\vn = \frac{\vvph}{\norm{\vvph}}$ and angle $\theta = \norm{\vvph}$. \\ \hline
Euler Angles ZYX &  $[z, y, x]^\transpose_{A\!B}$  & Tait-Bryan angles (Flight conv.): $z-y'-x''$, i.e.\   \\
Euler Angles YPR &  & yaw-pitch-roll. Singularities are at $y=\pm\frac{\pi}{2}$. \\
 &  & $z\in[-\pi,\pi), y\in[-\frac{\pi}{2},\frac{\pi}{2}), x\in[-\pi,\pi)$  \\  \hline
Euler Angles XYZ &  $[x, y, z]^\transpose_{A\!B}$ & Cardan angles: $x-y'-z''$, i.e.\ roll-pitch-yaw. \\
Euler Angles RPY & &  Singularities are at $y=\pm\frac{\pi}{2}$.  \\
 &  & $x\in[-\pi,\pi), y\in[-\frac{\pi}{2},\frac{\pi}{2}), z\in[-\pi,\pi)$  \\  \hline
\end{tabular} % \multirow{2}{*}{}

\subsubsection{Rotation Quaternion}
A rotation quaternion is a Hamiltonian unit quaternion: \\
$\begin{aligned}Q &= q_0 + q_1 i + q_2 j + q_3 k \in \mathbb{H}, \quad q_i \in \mathbb{R}, 
i^2 = j^2=k^2 = ijk = -1, \quad \norm{Q}= \sqrt{q_0^2 + q_1^2 + q_2^2 + q_3^2} = 1 \\
\end{aligned}$   \\
Note that $\Quat{I}{B}$ and $-\Quat{I}{B}$ represent the same rotation, but not the same unit quaternion. \\
Rot. quaternion as tuple: $Q = (q_0, q_1, q_2, q_3) = (q_0, \imquatvec{q})$ with $\imquatvec{q} := (q_1, q_2, q_3)^\transpose$ \\
Rot. quaternion as vector: $\vq = \begin{bmatrix} q_0 & q_1 & q_2 & q_3 \end{bmatrix}^\transpose$ \\
Conjugate: $Q^\ast = (q_0, -\imquatvec{q})$ \\
Inverse: $Q^{-1} = Q^\ast = (q_0, -\imquatvec{q})$ \\
%  The Eq. 109 and Eq. 111 in J. Diebel, 'Representating Attitude: Euler Angles, Unit Quaternions, and Rostation Vectors', Standford University, 2006. are wrong, they should be switched.
Quaternion multiplication: $\begin{aligned}Q \cdot P &= (q_0, \imquatvec{q})\cdot(p_0, \imquatvec{p}) = (q_0 p_0 - \imquatvec{q}^\transpose \imquatvec{p}, q_{0} \imquatvec{p} + p_0 \imquatvec{q} + \imquatvec{q} \times \imquatvec{p}) \quad \Leftrightarrow \\
	   \vq \otimes \vp &= \underbrace{\vQ(\vq)}_{\mathclap{\text{quaternion matrix}}}\vp \, = \, \begin{pmatrix}q_0 & -\imquatvec{q}^\transpose \\  \imquatvec{q} & q_0\identity_{3 \times 3}+\hat{\imquatvec{q}} \\ \end{pmatrix} \begin{pmatrix} p_0 \\ p_1 \\ p_2 \\ p_3 \end{pmatrix} 
	  = \begin{pmatrix} 
 q_0 & -q_1 & -q_2 & -q_3 \\
 q_1 &  q_0 & -q_3 &  q_2 \\
 q_2 &  q_3 &  q_0 & -q_1 \\
 q_3 & -q_2 &  q_1 &  q_0 \\ \end{pmatrix}\begin{pmatrix} p_0 \\ p_1 \\ p_2 \\ p_3 \end{pmatrix} \\
 &= \underbrace{\bar{\vQ}(\vp)}_{\mathclap{\text{conjugate quat. matrix}}}\vq \, = \, \begin{pmatrix}p_0 & -\imquatvec{p}^\transpose \\  \imquatvec{p} & p_0\identity_{3 \times 3}-\hat{\imquatvec{p}} \\ \end{pmatrix} \begin{pmatrix} q_0 \\ q_1 \\ q_2 \\ q_3 \end{pmatrix} 
	  = \begin{pmatrix} 
 p_0 & -p_1 & -p_2 & -p_3 \\
 p_1 &  p_0 & p_3 &  -p_2 \\
 p_2 &  -p_3 &  p_0 & p_1 \\
 p_3 & p_2 &  -p_1 &  p_0 \\ \end{pmatrix}\begin{pmatrix} q_0 \\ q_1 \\ q_2 \\ q_3 \end{pmatrix} \\
\end{aligned}$ \\


\subsubsection{Rotation Quaternion $\Leftrightarrow$ Rotation Angle-Axis}

\begin{tabular}{@{}lll@{}} 
$ \quat{I}{B} = \begin{bmatrix} \cos{\frac{\theta}{2}} \\ \vn \sin{\frac{\theta}{2}} \end{bmatrix}$ & $\Leftrightarrow$ & $
(\theta, \vn)_{I\!B} = \left\{
  \begin{array}{l l}
    (2\arccos{(p_0)}, \frac{\imquatvec{q}}{\norm{\imquatvec{q}}}) & \quad \text{if $\lVert \imquatvec{q} \rVert^2 \geq \epsilon^2$}\\
    (0, \begin{bmatrix}1 & 0 & 0\end{bmatrix}^\transpose)  & \quad \text{otherwise}
  \end{array} \right. $ \\
\end{tabular}
\subsubsection{Rotation Quaternion $\Leftrightarrow$ Direction Cosine Matrix }
$\begin{aligned}\comat{I}{B} &=  \identity_{3\times 3} + 2 q_0\hat{\imquatvec{q}} + 2 \hat{\imquatvec{q}}^2  = (2 q_0^2 -1) \identity_{3\times 3} + 2q_0\hat{\imquatvec{q}} + 2 \imquatvec{q}\imquatvec{q}^\transpose \\
&= \begin{bmatrix}
 p_0^2 + p_1^2 - p_2^2 - p_3^2 &         2p_1p_2 - 2p_0p_3 &         2p_0p_2 + 2p_1p_3 \\
         2p_0p_3 + 2p_1p_2 & p_0^2 - p_1^2 + p_2^2 - p_3^2 &         2p_2p_3 - 2p_0p_1 \\
         2p_1p_3 - 2p_0p_2 &         2p_0p_1 + 2p_2p_3 & p_0^2 - p_1^2 - p_2^2 + p_3^2 \\ \end{bmatrix} \\
\end{aligned}$

$\begin{aligned}\comat{I}{B}^{-1} &= \comat{B}{I}  = \identity_{3 \times 3} - 2 p_0\hat{\imquatvec{p}} + 2 \hat{\imquatvec{p}}^2 \\
&= \begin{bmatrix}  p_0^2 + p_1^2 - p_2^2 - p_3^2 &         2p_0p_3 + 2p_1p_2 &         2p_1p_3 - 2p_0p_2 \\
         2p_1p_2 - 2p_0p_3 & p_0^2 - p_1^2 + p_2^2 - p_3^2 &         2p_0p_1 + 2p_2p_3 \\
         2p_0p_2 + 2p_1p_3 &         2p_2p_3 - 2p_0p_1 & p_0^2 - p_1^2 - p_2^2 + p_3^2 \\ \end{bmatrix}  \\  
\end{aligned}$ \\

\subsubsection{Euler Angles ZYX $\Leftrightarrow$ Direction Cosine Matrix}
\myfigure{\includegraphics[width=1\columnwidth]{coordinate_systems/coordinate_system_ypr_kindr1-crop.pdf}\vspace{-3mm}
\figcaption{Rotation from $A$-frame to $D$-frame: ($z-y'-x''$) -- (yaw-pitch-roll) -- ($50^\circ-25^\circ-30^\circ)$}\label{fig:yaw-pitch-roll}}
\subsubsection{Euler Angles XYZ $\Leftrightarrow$ Direction Cosine Matrix}
\myfigure{\includegraphics[width=1\columnwidth]{coordinate_systems/coordinate_system_rpy_kindr1-crop.pdf}\vspace{-3mm}
\figcaption{Rotation from $A$-frame to $D$-frame: ($x-y'-z''$) -- (roll-pitch-yaw) -- ($50^\circ-25^\circ-30^\circ)$}\label{fig:roll-pitch-yaw}}


\subsection{Pose}

\subsubsection{Homogeneous Transformation Matrix}
 $\begin{bmatrix}{}_I\vr_{I\!P} \\ 1 \end{bmatrix} = \vT_{I\!B} \begin{bmatrix}{}_B\vr_{B\!P}\\ 1 \end{bmatrix}, \quad \vT_{I\!B} = \begin{bmatrix}
              \comat{I}{B} & \pos{I}{I}{B} \\
	      \zero^\transpose & 1 \\
             \end{bmatrix}
\quad
 \vT_{I\!B}^{-1} =\vT_{B\!I} = \begin{bmatrix}
              \comat{I}{B}^\transpose & -\comat{I}{B}^\transpose \pos{I}{I}{B} \\
	      \zero^\transpose & 1 \\
             \end{bmatrix}
             $

\section{Time Derivatives of Position \& Orientation}

\subsection{Linear Velocity}
Velocity of point $P$ expressed in a rotating frame $B$  w.r.t. to the inertial frame $I$: \\
${}_B\vv_P = {}_B\vv_A  + {}_B\dot{\vr}_{AP} + \rotvel{B}{I}{B} \times \pos{B}{A}{P}$ \\
Velocity of point $Q$ on rigid body $B$ that rotates with ${}_B\vOm$, where point $P$ is on the same rigid body $B$: \\
${}_B\vv_Q = {}_B\vv_{P} + {}_B\vOm \times \pos{B}{P}{Q}, \quad {}_B\vOm=\rotvel{B}{I}{B}$ 

\subsection{Angular Velocity}

\begin{tabular}{@{}ll@{}}
$\rotvel{B}{I}{B} =: {}_B\vOm$ & (local) absolute angular velocity of rigid body $B$ expr. in frame $B$ \\ 
$\rotvel{B}{I}{B} = -\rotvel{B}{B}{I}$ & inverse of angular velocity \\
$\rotvel{I}{I}{B} =  \comat{I}{B} \rotvel{B}{I}{B}$  &  (global) angular velocity from frame $B$ to frame $I$ \\
$\rotvelhat{I}{I}{B} =  \comat{I}{B} \rotvelhat{B}{I}{B} \comat{I}{B}^\transpose$ & coord. transformation of angular velocity from frame $B$ to frame $I$  \\
$\rotvel{D}{A}{D} = \rotvel{D}{A}{B} +  \rotvel{D}{B}{C} + \rotvel{D}{C}{D}$ & composition of (relative) angular velocity \\
\end{tabular}

\subsubsection{Time Derivative of Direction Cosine Matrix $\Leftrightarrow$ Angular Velocity}
\begin{tabular}{@{}lll@{}}
$\rotvelhat{I}{I}{B} = \dcomat{I}{B}\comat{I}{B}^\transpose  = \dcomat{B}{I}^\transpose\comat{B}{I}$  & $\Leftrightarrow$ & $\dcomat{I}{B} = \rotvelhat{I}{I}{B}\comat{I}{B}$ \\
$\rotvelhat{B}{I}{B} = \comat{I}{B}^\transpose \dcomat{I}{B} = \comat{B}{I} \dcomat{B}{I}^\transpose$ & $\Leftrightarrow$ & $\dcomat{I}{B} = \comat{I}{B}\rotvelhat{B}{I}{B}$ \\
\end{tabular}

\subsubsection{Time Derivative of Rotation Quaternion $\Leftrightarrow$ Angular Velocity}
\begin{tabular}{@{}lll@{}}
%  The following line is correct  according to Eq. 164 and Eq. 158 in J. Diebel, 'Representating Attitude: Euler Angles, Unit Quaternions, and Rostation Vectors', Standford University, 2006.
$\rotvel{I}{I}{B} = 2 \vH(\quat{I}{B}) \dot{\vq}_{I\!B}$  & $\Leftrightarrow$ &  $\dot{\vq}_{I\!B} = \frac{1}{2}\vH(\quat{I}{B})^\transpose \rotvel{I}{I}{B}$ \\ 
%  The following line correct  according to Eq. 165 and Eq. 159 in J. Diebel, 'Representating Attitude: Euler Angles, Unit Quaternions, and Rostation Vectors', Standford University, 2006.
$\rotvel{B}{I}{B} = 2 \bar{\vH}(\quat{I}{B}) \dot{\vq}_{I\!B}$  &  $\Leftrightarrow$ & $\dot{\vq}_{I\!B} = \frac{1}{2}\bar{\vH}(\quat{I}{B})^\transpose \rotvel{B}{I}{B}$   \\ 
%  The following line correct  according to Eq. 150 and Eq. 151 in J. Diebel, 'Representating Attitude: Euler Angles, Unit Quaternions, and Rostation Vectors', Standford University, 2006.
$\begin{aligned}\vH(\vq) &= \begin{bmatrix}-\imquatvec{q} & \hat{\imquatvec{q}}+q_0\identity_{3\times 3}\end{bmatrix} \in \mathbb{R}^{3\times4} \\
 &=\begin{bmatrix}  -q_1 &  q_0 & -q_3 &  q_2 \\
 -q_2 &  q_3 &  q_0 & -q_1 \\
 -q_3 & -q_2 &  q_1 &  q_0 \end{bmatrix}
\end{aligned}$ & &  $\begin{aligned}\bar{\vH}(\vq) &= \begin{bmatrix}-\imquatvec{q} & -\hat{\imquatvec{q}}+q_0\identity_{3\times 3}\end{bmatrix} \in \mathbb{R}^{3\times4} \\
 &=\begin{bmatrix} -q_1 &  q_0 &  q_3 & -q_2 \\
 -q_2 & -q_3 &  q_0 &  q_1 \\
 -q_3 &  q_2 & -q_1 &  q_0 \end{bmatrix}
\end{aligned}$\\
\end{tabular}

\subsubsection{Time Derivative of Angle-Axis  $\Leftrightarrow$ Angular Velocity}
\begin{tabular}{@{}ll@{}}
$\rotvel{I}{I}{B} = \vn \dot{\theta} + \dot{\vn}\sin{\theta} + \hat{\vn}\dot{\vn}(1-\cos{\theta})$ &  \\
$\rotvel{B}{I}{B} = \vn \dot{\theta} + \dot{\vn}\sin{\theta} - \hat{\vn}\dot{\vn}(1-\cos{\theta})$ &   \\
$\dot{\theta} = \vn^\transpose \rotvel{I}{I}{B}, \quad \dot{\vn}=\left(-\frac{1}{2}\frac{\sin{\theta}}{1-cos{\theta}}\hat{\vn}^2 -\frac{1}{2}\hat{\vn}\right) \rotvel{I}{I}{B} \quad \forall \theta \in \mathbb{R}\backslash\{0\}$  & \\
$\dot{\theta} = \vn^\transpose \rotvel{B}{I}{B}, \quad \dot{\vn}=\left(-\frac{1}{2}\frac{\sin{\theta}}{1-cos{\theta}}\hat{\vn}^2 +\frac{1}{2}\hat{\vn}\right) \rotvel{B}{I}{B} \quad \forall \theta \in \mathbb{R}\backslash\{0\}$  & \\
\end{tabular}

\subsubsection{Time Derivative of Rotation Vector $\Leftrightarrow$ Angular Velocity}
\begin{tabular}{@{}ll@{}}
$\rotvel{I}{I}{B} = \left(\identity_{3\times 3} + \hat{\vvph}\left(\frac{1-\cos{\norm{\vvph}}}{\norm{\vvph}^2}\right) +\hat{\vvph}^2 \left( \frac{\norm{\vvph}-\sin{\norm{\vvph}}}{\norm{\vvph}^3} \right)\right)\dot{\vvph} \quad \forall \norm{\vvph} \in \mathbb{R}\backslash\{0\}$ & \\
$\rotvel{B}{I}{B} = \left(\identity_{3\times 3} - \hat{\vvph}\left(\frac{1-\cos{\norm{\vvph}}}{\norm{\vvph}^2}\right) +\hat{\vvph}^2 \left( \frac{\norm{\vvph}-\sin{\norm{\vvph}}}{\norm{\vvph}^3} \right)\right)\dot{\vvph} \quad \forall \norm{\vvph} \in \mathbb{R}\backslash\{0\}$ & \\
$\dot{\vvph} = \left(\identity_{3\times 3} - \frac{1}{2}\hat{\vvph} + \hat{\vvph}^2 \frac{1}{\norm{\vvph}^2}\left(1 - \frac{\norm{\vvph}}{2}\frac{\sin{\norm{\vvph}}}{1-\cos{\norm{\vvph}}}\right)\right)\rotvel{I}{I}{B} \quad \forall \norm{\vvph} \in \mathbb{R}\backslash\{0\}$ & \\
$\dot{\vvph} = \left(\identity_{3\times 3} + \frac{1}{2}\hat{\vvph} + \hat{\vvph}^2 \frac{1}{\norm{\vvph}^2}\left(1 - \frac{\norm{\vvph}}{2}\frac{\sin{\norm{\vvph}}}{1-\cos{\norm{\vvph}}}\right)\right)\rotvel{B}{I}{B} \quad \forall \norm{\vvph} \in \mathbb{R}\backslash\{0\}$ & \\
\end{tabular}

\subsubsection{Time Derivative of Euler Angles ZYX  $\Leftrightarrow$ Angular Velocity}

% Project B_w_IB to dZYX
$\begin{bmatrix} \dot{z} \\ \dot{y} \\ \dot{x}  \end{bmatrix} = 
\begin{bmatrix}
0 			& \dfrac{\sin(x)}{\cos(y)} 			& \dfrac{\cos(x)}{\cos(y)}\\
0 			& \cos(x)        					& -\sin(x) \\
1 			& \dfrac{\sin(x)\sin(y)}{\cos(y)} 	& \dfrac{\cos(x)\sin(y)}{\cos(y)}
\end{bmatrix}
\rotvel{B}{I}{B}
\quad
\forall y \in \mathbb{R} \backslash \{ \frac{\pi}{2}+k\pi \},k\in \mathbb{Z}
$


% Project I_w_IB to dZYX
$\begin{bmatrix} \dot{z} \\ \dot{y} \\ \dot{x}  \end{bmatrix} = 
\begin{bmatrix}
\dfrac{\cos(z)\sin(y)}{\cos(y)}		&		\dfrac{\sin(y)\sin(z)}{\cos(y)}	&	1 \\
-\sin(z)							&		\cos(z)							&	0 \\
\dfrac{\cos(z)}{\cos(y)}			&		\dfrac{\sin(z)}{\cos(y)}		&	0 \\
\end{bmatrix}
\rotvel{I}{I}{B}
\quad
\forall y \in \mathbb{R} \backslash \{ \frac{\pi}{2}+k\pi \},k\in \mathbb{Z}
$


% Project dZYX to B_w_IB
$ \rotvel{B}{I}{B} =
\begin{bmatrix}
-\sin(y) 		&		0			& 1 \\
\cos(y)\sin(x)	&		\cos(x)		& 0 \\
\cos(x)\cos(y)	&	   -\sin(x)  	& 0 \\
\end{bmatrix}
\begin{bmatrix} \dot{z} \\ \dot{y} \\ \dot{x}  \end{bmatrix} $


% Project dZYX to I_w_IB
$ \rotvel{I}{I}{B} =
\begin{bmatrix}
0	&	-\sin(z)	&	 \cos(y)\cos(z)	\\
0	&	 \cos(z)	&	 \cos(y)\sin(z)	\\
1	&	 0			&	-\sin(y)
\end{bmatrix}
\begin{bmatrix} \dot{z} \\ \dot{y} \\ \dot{x}  \end{bmatrix} $


\newpage
\section{Dynamics of a Multi-Rigid-Body System}

\begin{tabular}{@{}ll@{}}
  $n$			& Number of bodies in system \\
  $n_j$ 		& Number of DoFs of the joints \\
  $n_q$ 		& Number of generalized coordinates \\
  $n_u$ 		& Number of generalized velocities \\
  $\vM$			& Mass matrix \\
  $\vg$			& Gyroscopic and Coriolis forces \\
  $\vf$			& Generalized external forces and torques \\
  $\vh$			& Combined force vector \\
  $\vJ_P$		& Jacobi matrix for translation of point P \\
  $\vJ_{R}$		& Jacobi matrix for rotation \\
  $\vF_Q^A$		& External forces on point Q \\
  $\vM^A$		& External torques \\
%   $\vp$			& Momentum \\
%   $\vN_S$		& Spin at center of gravity \\
  $m$ 			& Mass \\
  $\vTh$ 		& Inertia tensor \\
%   $\vv_P$		& Velocity of point P \\
%   $\va_P$		& Acceleration of point P \\
%   $\vOm$		& Angular velocity \\
%   $\vPs$		& Angular acceleration \\
  $(...)^-$		& Variable before impact \\
  $(...)^+$		& Variable after impact \\
  $(...)^\pm$		& Variable before/after impact \\
  $\mathrm{\Delta} t$	& Time step duration \\
  $\mathrm{\Delta} \vu$	& Velocity change over one time step \\
  $\vW$			& Generalized force directions for contact forces \\
  $\vla$		& Lebesgue-measurable contact forces \\
  $\vLa$		& Purely atomic impact impulses \\
  $\vP$			& Contact percussions \\
\end{tabular}


\subsection{Generalized Coordinates of a Floating-Base System with Rotational Joints}
Recommended set of generalized coordinates $\vq$ with quaternion $\quat{I}{B}$ and generalized velocities $\vu$:
\begin{tabular}{@{}lll@{}}
$\vq = \begin{pmatrix} \pos{I}{I}{B} \\ \quat{I}{B} \\ \varphi_1 \\ \vdots \\ \varphi_{n_j} \end{pmatrix} \in \mathbb{R}^{7+n_j} = \mathbb{R}^{n_q}$ & 
$\vu = \begin{pmatrix} {}_I \vv_B \\ \rotvel{B}{I}{B} \\ \dot{\varphi}_1 \\ \vdots \\ \dot{\varphi}_{n_j} \end{pmatrix} \in \mathbb{R}^{6+n_j} = \mathbb{R}^{n_u}$  & 
$\dot{\vu} = \begin{pmatrix} {}_I \va_B \\ {}_B\vps_{I\!B} \\ \ddot{\varphi}_1 \\ \vdots \\ \ddot{\varphi}_{n_j} \end{pmatrix} \in \mathbb{R}^{6+n_j}$
\\
\end{tabular}
$\dot{\vq} = \vF \vu, \quad \vF = \begin{pmatrix} \identity_{3\times 3} & \mathbf{0} & \mathbf{0} \\ \mathbf{0} & \frac{1}{2}\bar{\vH}^\mathsf{T} & \mathbf{0} \\ \mathbf{0} & \mathbf{0} & \identity_{n_j\times n_j} \end{pmatrix} \quad \Leftrightarrow \quad \vu = \bar{\vF}\dot{\vq}, \quad \bar{\vF} = \begin{pmatrix} \identity_{3x3} & \mathbf{0} & \mathbf{0} \\ \mathbf{0} & 2\bar{\vH} & \mathbf{0} \\ \mathbf{0} & \mathbf{0} & \identity_{n_j \times n_j} \end{pmatrix}$ \\

\subsection{Equations of Motion with Contacts and no Impulses}
Projected Newton-Euler Equations
\begin{tabular}{@{}ll@{}}
$\boxed{\vM\dot{\vu} - \vh = \vW\vla}$ with $\vh := \vf - \vg$, and &
$\begin{aligned}\vM &= \sum_{i=1}^n \left[ (\vJ_S^\mathsf{T} m \vJ_S + \vJ_R^\mathsf{T} \vTh_S \vJ_R) \right]_i \\
\vg &= \sum_{i=1}^n \left[ (\vJ_S^\mathsf{T} m \dot{\vJ}_S \vu + \vJ_R^\mathsf{T} (\vTh_S \dot{\vJ}_R \vu + \vOm \times \vTh_S \vOm)) \right]_i \\
\vf &= \sum_{i=1}^n \left[ (\vJ_Q^\mathsf{T} \vF_Q^A + \vJ_R^\mathsf{T} \vM^A) \right]_i \end{aligned}$ \\
\end{tabular}

\subsection{Equations of Motion with Contacts and Impulses}
$\boxed{\vM \mathrm{\Delta} \vu - \vh \mathrm{\Delta} t = \vW \vP}  \quad \left\{
  \begin{array}{r l} \vM (\vu^+ - \vu^-) &= \vW \vLa \\
 \vM \underbrace{(\dot{\vu} \mathrm{d}t + (\vu^+ - \vu^-) \mathrm{d}\eta)}_{\mathrm{d}\vu} - \vh \mathrm{d}t &= \vW \underbrace{(\vla \mathrm{d}t + \vLa \mathrm{d}\eta)}_{\mathrm{d}\vP}\end{array}\right.$ \\

\subsection{Transformation of Equations of Motion}
Transformation from  $\bar{\vM}(\bar{\vq}),\bar{\vh}(\bar{\vq},\bar{\vu})$ to $\vM(\vq),\vh(\vq, \vu)$,
where $\bar{\vu} = \vB \vu$:
$\begin{aligned}
\vM &= \vB^\mathsf{T} \bar{\vM} \vB \\
\vh &= \vB^\mathsf{T} \bar{\vh} - \vB^\mathsf{T} \bar{\vM} \dot{\vB} \vu \\
\end{aligned}$


%%%%%%%%%%%%%%%%%%%%%%%%%%%%%%%%%%%%%%%%%%%%%%%%%%%%%%%%%%%%%%%%%%%%%%%%%%%%%%%%%%%%%%%
%  END CONTENT
%%%%%%%%%%%%%%%%%%%%%%%%%%%%%%%%%%%%%%%%%%%%%%%%%%%%%%%%%%%%%%%%%%%%%%%%%%%%%%%%%%%%%%%
%\rule{0.3\linewidth}{0.25pt}
%\scriptsize

%Copyright \copyright\ 2014 Autonomous Systems Lab, ETH Zurich (asl.ethz.ch)\\
%Contact: Christian Gehring (gehrinch@ethz.ch) \\

\end{multicols}
\end{document}
